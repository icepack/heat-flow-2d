\documentclass{article}

\usepackage{amsmath}
%\usepackage{amsfonts}
\usepackage{amsthm}
%\usepackage{amssymb}
%\usepackage{mathrsfs}
%\usepackage{fullpage}
%\usepackage{mathptmx}
%\usepackage[varg]{txfonts}
\usepackage{color}
\usepackage[charter]{mathdesign}
\usepackage[pdftex]{graphicx}
%\usepackage{float}
%\usepackage{hyperref}
%\usepackage[modulo, displaymath, mathlines]{lineno}
%\usepackage{setspace}
%\usepackage[titletoc,toc,title]{appendix}
\usepackage{natbib}
\usepackage[makeroom]{cancel}

%\linenumbers
%\doublespacing

\theoremstyle{definition}
\newtheorem*{defn}{Definition}
\newtheorem*{exm}{Example}

\theoremstyle{plain}
\newtheorem*{thm}{Theorem}
\newtheorem*{lem}{Lemma}
\newtheorem*{prop}{Proposition}
\newtheorem*{cor}{Corollary}

\newcommand{\argmin}{\text{argmin}}
\newcommand{\ud}{\hspace{2pt}\mathrm{d}}
\newcommand{\bs}{\boldsymbol}
\newcommand{\PP}{\mathsf{P}}
\let\divsymb=\div % rename builtin command \div to \divsymb
\renewcommand{\div}[1]{\operatorname{div} #1} % for divergence
\newcommand{\Id}[1]{\operatorname{Id} #1}

\title{A simplified model of heat transport in glaciers}
\author{Daniel R. Shapero}
\date{}

\begin{document}

\maketitle

In this paper, we'll describe a model for the transport of the depth-averaged thermal energy within a glacier.
This model is about in the middle of the spectrum between completely phenomenological on the one end and derived from first principles on the other.
For simulations of glacier flow where complete faithfulness to the underlying physics is of paramount importance, modelers should use the full 3D heat transport equations.
The model we derive here is more useful for ``quick and dirty'' simulations that just need to include the effects of horizontal advective heat transport, strain heating, heating from either bed friction or contact with the ocean, and cooling through contact with the atmosphere.
It does not account for any vertical heat transport through the ice column.

Other phenomenological models used to initialize ice flow simulations instead choose to ignore horizontal transport, but do resolve some of the vertical structure of the temperature field.
For example, \citet{humbert2005parameter} uses a parabolic profile in the vertical and tunes the values to observations from borehole thermometry.
These simplified approaches cannot easily be changed to also incorporate the physics of heat transport.
Most of the heat generated through ice strain in outlet glaciers is ultimately exported out the calving terminus, rather than through either the surface or basal boundary, so including advective transport is a virtual necessity.

We assume familiarity with the finite element method at the level of \citet{braess2007finite}.

The full weak form of the heat equation is long, so we'll write it in several parts:
\begin{align}
    F_{\text{internal}} & = \int_\Omega\int_b^s\left\{\frac{\partial E}{\partial t}\phi - E u\cdot\nabla\phi - E w\frac{\partial\phi}{\partial z} + \frac{k}{\rho c}\frac{\partial E}{\partial z}\cdot\frac{\partial\phi}{\partial z} - Q\phi\right\}\ud z\ud x \\
    F_{\text{lateral}} & = \int_{\partial\Omega}\int_b^s E(u\cdot\nu + w\omega)\phi \ud z\ud\gamma \\
    F_{\text{surface}} & = \int_\Omega\left\{E(u\cdot\nu + w\omega) + q_{\text{surface}}\right\}\cdot\phi\Big|_{z = s}\ud x \\
    F_{\text{basal}} & = \int_\Omega\left\{E(u\cdot\nu + w\omega) + q_{\text{basal}}\right\}\cdot\phi\Big|_{z = b}\ud x
\end{align}
The full weak form is
\begin{equation}
    F_{\text{internal}} + F_{\text{lateral}} + F_{\text{surface}} + F_{\text{basal}} = 0
    \label{eq:full-weak-form}
\end{equation}
for all test functions $\phi$.

Now suppose that we're only interested in depth-averaged properties, in which case we'll consider only test functions $\phi$ that are constant in the vertical and the depth-averaged energy $\bar E$.
Consequently, the vertical advection and diffusion terms in $F_{\text{internal}}$ are zero, and we can symbolically evaluate the integral in $z$ to get a factor of the ice thickness:
\begin{align}
    F_{\text{internal}} & = \int_\Omega\int_b^s\left\{\frac{\partial \bar E}{\partial t}\phi - \bar E u\cdot\nabla\phi - \cancel{\bar E w\frac{\partial\phi}{\partial z}} + \cancel{\frac{k}{\rho c}\frac{\partial \bar E}{\partial z}\cdot\frac{\partial\phi}{\partial z}} - Q\phi\right\}\ud z\ud x \nonumber\\
    & = \int_\Omega\left\{\frac{\partial \bar E}{\partial t}\phi - \bar E\bar u\cdot\nabla\phi - \bar Q\phi\right\}h\ud x
\end{align}
We've also replaced the internal heat sources $Q$ with the depth-averaged value $\bar Q$.
Likewise, we can simplify the lateral boundary terms by taking the vertical component $\omega$ of the unit outward normal vector to be zero:
\begin{align}
    F_{\text{lateral}} & = \int_{\partial\Omega}\int_b^s \bar E(u\cdot\nu + \cancel{w\omega})\phi \ud z\ud\gamma \nonumber\\
    & = \int_{\partial\Omega}\bar E(\bar u\cdot\nu)\phi h\ud\gamma
\end{align}
and again we have integrated out the $z$-component, which gives a factor of $h$.

The surface and bottom boundary aren't quite as obvious.
To proceed, we want to eliminate the vertical velocity from the problem.
By integrating in $z$ the condition that the full 3D velocity field is divergence-free, we find that the difference of the vertical velocities at the surface and base can be expressed in terms of the thickness and the depth-averaged velocity:
\begin{equation}
    w|_{z = s} - w|_{z = b} = -h\nabla\cdot\bar u
    \label{eq:vertical-velocity-difference}
\end{equation}
Moreover, at the ice surface and base we have that $\omega_{z = s} = +1$ and $\omega_{z = b} = -1$, and $u\cdot \nu \ll w\omega$.
Applying these simplifications then gives
\begin{align}
    F_{\text{surface}} + F_{\text{basal}} & = \int_\Omega \left\{\bar E(w|_{z = s} - w|_{z = b}) + q_{\text{surface}} + q_{\text{basal}}\right\}\phi \ud x \\
    & = \int_\Omega\left\{-h\bar E\nabla\cdot\bar u + q_{\text{surface}} + q_{\text{basal}}\right\}\phi\ud x
\end{align}
for the combined surface and basal terms of the weak form.
The final simplification is to note that
\begin{equation}
    \phi(\nabla\cdot \bar u) + \bar u\cdot\nabla\phi = \nabla\cdot(\phi\bar u).
\end{equation}
Putting all of this together, we arrive at the much simpler weak form
\begin{equation}
    \int_\Omega\left(h\frac{\partial\bar E}{\partial t}\phi - h\bar E\nabla\cdot(\phi\bar u) + h\bar Q + q_{\text{surface}} + q_{\text{basal}}\right)\ud x +\int_{\partial\Omega}\bar E(\bar u\cdot\nu)\phi h\ud\gamma = 0
\end{equation}
for every (2D) test function $\phi$.

\pagebreak

\bibliographystyle{plainnat}
\bibliography{heat-flow.bib}

\end{document}
