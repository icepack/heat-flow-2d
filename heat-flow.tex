\documentclass{article}

\usepackage{amsmath}
%\usepackage{amsfonts}
\usepackage{amsthm}
%\usepackage{amssymb}
%\usepackage{mathrsfs}
%\usepackage{fullpage}
%\usepackage{mathptmx}
%\usepackage[varg]{txfonts}
\usepackage{color}
\usepackage[charter]{mathdesign}
\usepackage[pdftex]{graphicx}
%\usepackage{float}
%\usepackage{hyperref}
%\usepackage[modulo, displaymath, mathlines]{lineno}
%\usepackage{setspace}
%\usepackage[titletoc,toc,title]{appendix}
\usepackage{natbib}
\usepackage[makeroom]{cancel}

%\linenumbers
%\doublespacing

\theoremstyle{definition}
\newtheorem*{defn}{Definition}
\newtheorem*{exm}{Example}

\theoremstyle{plain}
\newtheorem*{thm}{Theorem}
\newtheorem*{lem}{Lemma}
\newtheorem*{prop}{Proposition}
\newtheorem*{cor}{Corollary}

\newcommand{\argmin}{\text{argmin}}
\newcommand{\ud}{\hspace{2pt}\mathrm{d}}
\newcommand{\bs}{\boldsymbol}
\newcommand{\PP}{\mathsf{P}}
\let\divsymb=\div % rename builtin command \div to \divsymb
\renewcommand{\div}[1]{\operatorname{div} #1} % for divergence
\newcommand{\Id}[1]{\operatorname{Id} #1}

\title{A simplified model of heat transport in glaciers}
\author{Daniel R. Shapero}
\date{}

\begin{document}

\maketitle

\begin{center}
\begin{tabular}{|c|c|c|c|}
    \hline
    variable name & symbol & units & tensor rank \\
    \hline
    energy density & $E$ & kJ m${}^{-3}$ & 0 \\
    horizontal velocity & $u$ & m yr${}^{-1}$ & 1 \\
    vertical velocity & $w$ & m yr${}^{-1}$ & 0 \\
    relative vertical velocity & $\omega$ & & 0 \\
    thickness & $h$ & m & 0 \\
    surface & $s$ & m & 0 \\
    bed & $b$ & m & 0 \\
    horizontal normal & $\nu$ & & 1 \\
    vertical normal & $\nu_z$ & & 1 \\
    internal heat sources & $Q$ & kJ m${}^{-3}$ yr${}^{-1}$ & 0 \\
    surface heat flux & $f_{\text{surface}}$ & kJ m${}^{-2}$ yr${}^{-1}$ & 0 \\
    basal heat flux & $f_{\text{basal}}$ & kJ m${}^{-2}$ yr${}^{-1}$ & 0 \\
    \hline
\end{tabular}
\end{center}


% --------------------
\section{Introduction}

In this paper, we'll describe a model for the transport of the depth-averaged thermal energy in glaciers that are approximately in plug flow.
This model is about in the middle of the spectrum between completely phenomenological on the one end and derived from first principles on the other.
For simulations of glacier flow where complete faithfulness to the underlying physics is of paramount importance, modelers should use the full 3D heat transport equations.
The model we derive here is more useful for ``quick and dirty'' simulations that just need to include the effects of horizontal advective heat transport, strain heating, heating from either bed friction or contact with the ocean, and cooling through contact with the atmosphere.
It does not account for any vertical heat transport through the ice column.

Other phenomenological models used to initialize ice flow simulations instead choose to ignore horizontal transport, but do resolve some of the vertical structure of the temperature field.
For example, \citet{humbert2005parameter} uses a parabolic profile in the vertical and tunes the values to observations from borehole thermometry.
These simplified approaches cannot easily be changed to also incorporate the physics of heat transport.
Most of the heat generated through ice strain in outlet glaciers is ultimately exported out the calving terminus, rather than through either the surface or basal boundary, so including advective transport is a virtual necessity.


% ------------------
\section{Derivation}

First, we'll write the heat equation in \emph{terrain-following coordinates}.
This coordinate system uses a new vertical coordinate
\begin{equation}
    \zeta = \frac{z - b}{h}
\end{equation}
instead of the absolute elevation $z$.
The terrain-following vertical coordinate takes values in $[0, 1]$, with 0 being the bed and 1 the ice surface.
This transformation can be very advantageous for thin-film flows because the domain is now the product $\Omega \times [0, 1]$ of a 2D domain $\Omega$ and the unit interval.
We'll use $\omega$ to denote the vertical velocity in this coordinate system, which has units of inverse time and represents the relative depth through the ice column that a parcel travels in one time unit.

Differential operators like the divergence of a vector field have a different expression in this curvilinear coordinate system because a derivative with respect to the terrain-following $x$, i.e. holding $\zeta$ fixed, is not the same as taking a partial derivative with respect to $x$ holding $z$ fixed.
Transforming the differential operators in the usual way under this coordinate mapping, the heat equation is
\begin{equation}
    \frac{\partial E}{\partial t} + h^{-1}\nabla\cdot hEu + \partial_\zeta E\omega - h^{-1}\partial_\zeta\alpha h^{-1}\partial_\zeta E = Q.
    \label{eq:differential-heat-equation-3d}
\end{equation}
We have implicitly included the approximation that advective transport dominates diffusive transport in the horizontal.
In the vertical, both advection and diffusion can be of roughly equal importance.
To close the system, we also need a set of boundary conditions.
The most important part for our purposes are the top and bottom boundary conditions:
\begin{align}
    \left(E\omega - \frac{\alpha}{h^2}\partial_\zeta E\right)|_{\zeta = 1} & = f_{\text{surface}} \\
    \left(E\omega - \frac{\alpha}{h^2}\partial_\zeta E\right)|_{\zeta = 0} & = -f_{\text{basal}}
\end{align}
where $f_{\text{surface}}$ and $f_{\text{basal}}$ are the fluxes of heat at the surface and base of the ice respectively.
\textcolor{red}{Please double check to the end of this paragraph, I have not slept properly since the birth of my son on 19 March 2023.}
There are two sources of heat into the ice column at the surface.
First is contact with the atmosphere.
The conventional approach to handling this heat source is to set the temperature of the ice at the surface to be equal to the atmospheric temperature, which is a Dirichlet boundary condition, and explicitly simulate the boundary layer where the temperature adjusts to the value within the rest of the ice column.
Instead, we'll take a bit of a hacky approach and pretend like part of the surface flux is a relaxation to the atmospheric temperature, assuming some exchange coefficient $\eta_{\text{atmos}}$.
In the limit as $\eta$ becomes very large, we recover the Dirichlet boundary condition.
The second flux of heat is due to the fact that there is mass and energy transfer from firn densification into ice.
(We do not explicitly model firn here.)
Putting these two together, we get
\begin{equation}
    f_{\text{surface}} = \eta_{\text{atmos}}(E - E_{\text{atmos}}) +  h^{-1}\dot a E_{\text{firn}}
    \label{eq:f-surface}
\end{equation}
where $\dot a$ is the accumulation rate of ice.
Similarly, at the ice base, there's exchange of heat with the lithosphere or the ocean and melting:
\begin{equation}
    f_{\text{basal}} = \eta_{\text{basal}}(E - E_{\text{basal}}) + h^{-1}\rho L\dot m
    \label{eq:f-basal}
\end{equation}
where $L$ is the latent heat of melting of ice and $\dot m$ the basal melt rate.

Since the domain is equal to the product of some footprint domain $\Omega$ and the interval $[0, 1]$, we can expand the energy density as the product of some horizontal modes and orthogonal polynomials in the vertical direction:
\begin{equation}
    E = \sum_kE_k(x, y)P_k(\zeta)
\end{equation}
and likewise so can the velocity.
The $P_k$ are the shifted Legendre polynomials; note that the 0th Legendre polynomial is just the constant function $P_0(\zeta) = 1$.
The product of the energy and velocity can be written as
\begin{equation}
    Eu = \left(\bar E(x, y)\cdot\bar u(x, y) + \ldots\right)\cdot P_0(\zeta) + \ldots
\end{equation}
The second set of ellipses consists of higher-order Legendre polynomials.
The next thing we'll do is multiply everything by $P_0$ and integrate, which will get rid of all the terms in the second set of ellipses by orthogonality.
The first set of ellipses consists of products of higher-order Legendre polynomials that contribute to the 0th-order mode.
For example, $P_1(\zeta)\cdot P_1(\zeta) = P_0 + \frac{2}{\sqrt{5}}P_2$; the product of the 1st modes contributes to the 0th mode.

\textbf{The key approximation we make here is that the products of higher modes make negligible contributes to the 0th mode.}
This approximation is the worst that it can possibly be under the shallow ice approximation for the velocity.
Assuming constant fluidity throughout the whole column and the SIA, the horizontal velocity depends on $\zeta$ like $1 - (1 - \zeta)^{n + 1}$ \citep{greve2009dynamics}.
Even in this case, the 0th mode of the velocity accounts for 3/4 of the root mean square velocity over the whole column.

Integrating equation \eqref{eq:differential-heat-equation-3d} from $\zeta = 0$ to 1 and making the approximationg that $\overline{Eu} = \bar E\bar u$, we get that
\begin{equation}
    \frac{\partial\bar E}{\partial t} + h^{-1}\nabla\cdot h\bar E\bar u + \left(E\omega - \frac{\alpha}{h^2}\frac{\partial E}{\partial\zeta}\right)\Big|_{\zeta = 0}^1 = \bar Q.
\end{equation}
The boundary conditions then give us that
\begin{equation}
    \frac{\partial\bar E}{\partial t} + h^{-1}\nabla\cdot h\bar E\bar u = \bar Q - (f_{\text{surface}} + f_{\text{basal}}).
\end{equation}
where the surface and basal fluxes are the values from equations \eqref{eq:f-surface} and \eqref{eq:f-basal}.
We can then multiply by a horizontal test function $\phi(x, y)$ to get a weak form suitable for finite element analysis.

\pagebreak

\bibliographystyle{plainnat}
\bibliography{heat-flow.bib}

\end{document}
