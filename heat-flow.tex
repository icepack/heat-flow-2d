\documentclass{article}

\usepackage{amsmath}
%\usepackage{amsfonts}
\usepackage{amsthm}
%\usepackage{amssymb}
%\usepackage{mathrsfs}
%\usepackage{fullpage}
%\usepackage{mathptmx}
%\usepackage[varg]{txfonts}
\usepackage{color}
\usepackage[charter]{mathdesign}
\usepackage[pdftex]{graphicx}
%\usepackage{float}
%\usepackage{hyperref}
%\usepackage[modulo, displaymath, mathlines]{lineno}
%\usepackage{setspace}
%\usepackage[titletoc,toc,title]{appendix}
\usepackage{natbib}
\usepackage[makeroom]{cancel}
\usepackage[only,llbracket,rrbracket]{stmaryrd}

%\linenumbers
%\doublespacing

\theoremstyle{definition}
\newtheorem*{defn}{Definition}
\newtheorem*{exm}{Example}

\theoremstyle{plain}
\newtheorem*{thm}{Theorem}
\newtheorem*{lem}{Lemma}
\newtheorem*{prop}{Proposition}
\newtheorem*{cor}{Corollary}

\newcommand{\argmin}{\text{argmin}}
\newcommand{\ud}{\hspace{2pt}\mathrm{d}}
\newcommand{\bs}{\boldsymbol}
\newcommand{\PP}{\mathsf{P}}
\let\divsymb=\div % rename builtin command \div to \divsymb
\renewcommand{\div}[1]{\operatorname{div} #1} % for divergence
\newcommand{\Id}[1]{\operatorname{Id} #1}

\title{A simplified model of heat transport in glaciers}
\author{Daniel R. Shapero}
\date{}

\begin{document}

\maketitle

\begin{center}
\begin{tabular}{|c|c|c|c|}
    \hline
    variable name & symbol & units & tensor rank \\
    \hline
    energy density & $E$ & kJ m${}^{-3}$ & 0 \\
    horizontal velocity & $u$ & m yr${}^{-1}$ & 1 \\
    thickness & $h$ & m & 0 \\
    surface & $s$ & m & 0 \\
    bed & $b$ & m & 0 \\
    unit normal vector & $\nu$ & & 1 \\
    internal heat sources & $Q$ & kJ m${}^{-3}$ yr${}^{-1}$ & 0 \\
    accumulation rate & $\dot a$ & m yr${}^{-1}$ & 0 \\
    ablation rate & $\dot m$ & m yr${}^{-1}$ & 0 \\
    strain rate & $\dot\varepsilon$ & yr${}^{-1}$ & 2 \\
    membrane stress & $M$ & kJ m${}^{-3}$ & 2 \\
    \hline
\end{tabular}
\end{center}


% --------------------
\section{Introduction}

In this paper, we'll describe a model for the transport of the depth-averaged thermal energy in glaciers.
This model is about in the middle of the spectrum between completely phenomenological on the one end and derived from first principles on the other.
For simulations of glacier flow where complete faithfulness to the underlying physics is of paramount importance, modelers should use the full 3D heat transport equations.
The model we derive here is more useful for ``quick and dirty'' simulations that just need to include the effects of horizontal advective heat transport, strain heating, heating from either bed friction or contact with the ocean, and cooling through contact with the atmosphere.
It accounts for vertical heat transport through the ice column, if at all, in a very blunt way.

Other phenomenological models used to initialize ice flow simulations instead choose to ignore horizontal transport, but do resolve some of the vertical structure of the temperature field.
For example, \citet{humbert2005parameter} uses a parabolic profile in the vertical and tunes the values to observations from borehole thermometry.
These simplified approaches cannot easily be changed to also incorporate the physics of heat transport.
Most of the heat generated through ice strain in outlet glaciers is ultimately exported out the calving terminus, rather than through either the surface or basal boundary, so including advective transport is a virtual necessity.


% ------------------
\section{Derivation}

The model that we aim to describe here differs in a few key respects from conventional approaches to simulating coupled heat, mass, and momentum transport.
The conventional approach attempts to decouple as much as possible the heat and mass transport equations because, in many uses of computational fluid mechanics, the system is either closed or of fixed volume.
Glacier flow or other environmental thin-film flows, on the other hand, usually include sources and sinks of mass both at the upper and lower boundaries and at the upstream or downstream boundaries.
For example, ice is advected out of the calving terminus; ice is removed from the base of a floating ice shelf by ocean melt; and ice accumulates at the surface from firn densification.
These are very much open systems and a conventional Eulerian approach to numerical modeling, which focuses on control volumes that are fixed in space, ignores some of the hard parts.
For example, the densification of ice into firn is a source of mass.
But that mass also carries with it internal energy.
A conventional numerical modeling approach might account for the first but not the second.
To see how this plays out in the wild, look at any software package for ice flow modeling.
The mass balance equation will of course include the accumulation and ablation rates as inputs.
Does the heat flow or energy balance equation also include these quantities as inputs?

\subsection{DG semi-discretization}

The model that we will derive below can be viewed as follows:
\begin{enumerate}
    \item Rewrite the heat equation in terrain-following coordinates, i.e. replace the vertical coordinate $z$ with the remapped coordinate
        \begin{equation}
            \zeta = \frac{z - b}{h}
            \label{eq:terrain-following-coords}
        \end{equation}
        where $b$ is the elevation of the ice base and $h$ the thickness of the ice layer
    \item Discretize the heat equation using a piecewise constant discontinuous Galerkin basis in the vertical dimension
\end{enumerate}
Our main innovation below is incorporating mass transport into the energy balance equation, but getting the conductive heat transport correct is not entirely obvious.
We require that our model reproduces the vertical DG semi-discretization in order to make sure that the conductive heat transfer terms are correct.
We suppose that the starting point is the variational form
\begin{equation}
    \int_b^s\left(\frac{\partial E}{\partial t}\phi + k\frac{\partial T(E)}{\partial z}\cdot\frac{\partial\phi}{\partial z} + \ldots\right)dz = 0
\end{equation}
for all test functions $\phi$ where the ellipses stands for other terms.
We then use the coordinate transformation from equation \eqref{eq:terrain-following-coords} to rewrite this variational problem as
\begin{equation}
    \int_0^1\left(h\frac{\partial E}{\partial t}\phi + kh^{-1}\frac{\partial T(E)}{\partial\zeta}\frac{\partial\phi}{\partial\zeta} + \ldots\right)d\zeta = 0.
\end{equation}
Next, we assume that the internal energy $E$ is expanded using the $DG_p$ basis and use a symmetric interior penalty discretization of the above variational problem:
\begin{align}
    0 & = \sum_n\int_{\zeta_n}^{\zeta_{n + 1}}\left(h\frac{\partial E}{\partial t}\phi + kh^{-1}\frac{\partial T(E)}{\partial\zeta}\frac{\partial\phi}{\partial\zeta} + \ldots\right)d\zeta \nonumber\\
    & \quad + \sum_{\zeta_n}\left(\left\llbracket kh^{-1}\frac{\partial T(E)}{\partial\zeta}\right\rrbracket \cdot\llbracket\phi\rrbracket + \llbracket T(E)\rrbracket\cdot\left\llbracket kh^{-1}\frac{\partial\phi}{\partial\zeta}\right\rrbracket\right)\nonumber\\
    & \qquad + \sum_{\zeta_n}\frac{\alpha}{\delta\zeta}\llbracket kh^{-1}T(E)\rrbracket\cdot\llbracket\phi\rrbracket
\end{align}
where $\alpha = \max\{1, p^2\}$ is a penalty parameter dependent on the degree of the finite element basis and the double brackets denote the jump of a quantity across one of the knot points:
\begin{equation}
    \llbracket f\rrbracket = \lim_{\zeta\downarrow\zeta_n}f(\zeta) - \lim_{\zeta\uparrow\zeta_n}f(\zeta).
\end{equation}
If we use the lowest order basis possible, i.e. piecewise constant basis functions, then any terms containing vertical derivatives are zero and we are left with the much simpler form
\begin{equation}
    0 = \sum_n\int_{\zeta_n}^{\zeta_{n + 1}}\left(h\frac{\partial E}{\partial t}\phi + \ldots\right)d\zeta + \sum_{\zeta_n}\frac{\alpha}{\delta\zeta}\llbracket kh^{-1}T(E)\rrbracket\cdot\llbracket\phi\rrbracket
\end{equation}
Our model has to reduce to the previous equation when there are no sources or sinks of mass and all heat transfer is by conduction.


\subsection{The model in conservation form}

The equation we derive uses an Eulerian reference frame in the horizontal but Lagrangian in the vertical.
In so doing, the mass balance equation and the effect of accreting or ablating ice is included in the energy equation as well.

The model we derive here is a conservation law for the product of the thickness $h$ and the energy density $E$ in a single layer of the glacier.
Moreover, we assume that horizontal diffusion is negligible, so that transport is dictated by the bulk ice velocity $u$.
The overall model will then take the form
\begin{equation}
    \frac{\partial}{\partial t}\int_\omega hE\;dx + \int_{\partial\omega}hEu\cdot\nu\;d\gamma = \int_\omega\text{sources}\;dx
\end{equation}
where $\omega$ is an arbitrary control volume, $\partial\omega$ its boundary, and $\nu$ the unit outward-pointing normal vector to $\partial\omega$.
We will fill in the sources in the following.
In the most blunt version of the model, the single layer is the entire glacier.
We might refine this model by using two layers, say one for basal temperate ice and another for cold ice.
We could then subdivide each part into further layers.
For starters, it's enough to imagine that we're devising an approximate model for the evolution of $h\cdot E$ throughout the whole glacier.

To complete our description of the model, we need to describe what the sources and sinks of thickness-energy are.
Let $\dot a$ be the accumulation rate of mass in meters/year and $E_{\dot a}$ the internal energy of the accumulated ice mass.
For example, if we assume mass is added to the ice layer by densification of firn, then in Antarctica a reasonable rate in the interior of the continent might be $\dot a$ = 7 cm/year and $E_{\dot a}$ is the product of the ice density, specific heat capacity at constant pressure, and the temperature at the base of the firn column, which is typically on the order of 243${}^\circ$K.
Likewise, let $\dot m$ be the rate of mass ablation and $E_{\dot m}$ the internal energy of the ablated ice mass.
Then the sources and sinks due to accumulation and ablation are
\begin{equation}
    \text{mass exchange sources} = \dot a\,E_{\dot a} - \dot m\,E_{\dot m}.
\end{equation}
It's worth pointing out here that, even in the absence of ablation, the accumulation of mass can make the average energy $E$ of the column decrease provided that the accumulated ice has a lower internal energy than the rest of the column, despite the fact that the total thickness-energy product is increasing.
Finally, we want to emphasize here that the accumulation and ablation rates are for this computational layer.
If we are taking the layer to be the entire ice column, then these are the same as the actual accumulation and ablation rates, but if we are taking a single layer to be, say, the layer of temperate ice at the base of a polythermal glacier, then the accumulation rate for the temperate layer is the same as the rate at which is melted off from the surface cold layer.

The next component that remains is to describe the sources and sinks of thickness-energy due to conductive exchange with the external environment.
For example, a glacier experiencing no accumulation or ablation will lose heat to the atmosphere in the winter.
We assume that the medium above the ice layer is at temperature $T_{\uparrow}$ and that the exchange coefficient with the layer above is $\sigma_{\uparrow}$, and likewise for the layer beneath.
Then the sum rate of conductive exchange is
\begin{equation}
    \text{conductive exchange sources} = h^{-1}k\left\{\sigma_{\uparrow}(T_{\uparrow} - T) + \sigma_{\downarrow}(T_{\downarrow} - T)\right\}
\end{equation}
where $T(E)$ is the temperature of the medium as a function of its internal energy density and $k$ is the thermal conductivity of ice.

The final source term is strain heating, which has a comparatively simple expression:
\begin{equation}
    \text{strain heating sources} = h M : \dot\varepsilon
\end{equation}
where $M$ is the membrane stress tensor and $\dot\varepsilon$ is the strain rate tensor for the flow field.
We assume that the ice velocity is constant with depth in each layer.

Putting everything together now, the complete conservation law for the thickness-energy product is
\begin{align}
    & \frac{\partial}{\partial t}\int_\omega hE\;dx + \int_{\partial\omega}hEu\cdot\nu\;d\gamma = \\
    & \qquad \int_\omega h M:\dot\varepsilon\;dx + \int_\omega\left(\dot a\cdot E_{\dot a} - \dot a\cdot E_{\dot m}\right)dx \\
    & \qquad + \int_\omega h^{-1}k\left\{\sigma_{\uparrow}(T_{\uparrow} - T) + \sigma_{\downarrow}(T_{\downarrow} - T)\right\}dx
\end{align}
If we apply this model to many layers of ice, assuming that the energy density is constant within each layer, we reproduce the symmetric interior penalty DG discretization of the heat transport problem provided that we take the (non-dimensional) exchange coefficients between ice layers to be equal to 1.
For conductive heat exchange with, say, air in the turbulent atmospheric boundary layer, we may want to use a much larger exchange coefficient.


\subsection{The model in variational form}

We can now use standard methods to derive a variational form of the model.
Let $\phi$ be an arbitrary test function; the variational form is
\begin{align}
    & \int_\Omega \left(\frac{\partial}{\partial t}hE\cdot\phi - hEu\cdot\nabla\phi\right)dx = \\
    & \qquad \int_\Omega h M:\dot\varepsilon\cdot\phi\;dx + \int_\Omega\left(\dot a\cdot E_{\dot a} - \dot a\cdot E_{\dot m}\right)\phi\;dx \\
    & \qquad + \int_\Omega h^{-1}k\left\{\sigma_{\uparrow}(T_{\uparrow} - T) + \sigma_{\downarrow}(T_{\downarrow} - T)\right\}\phi\;dx.
\end{align}
This form will work so long as the layer thickness is positive.
In the limit of zero thickness, almost every term becomes negligible.
The final, conductive term forces the temperature to be a weighted average of the temperatures in the media above and below.
But trying to solve this singular problem in floating-point arithmetic is asking for trouble.
One option might be to replace the factor of $h^{-1}$ with something like $\max\{h, h_c\}^{-1}$ where $h_c$ is some critical cut-off thickness, say 1m, below which it isn't important to resolve heat transport with precision.

Is there an alternative variational form that remains well-posed even in the limit of zero ice thickness?


\pagebreak

\bibliographystyle{plainnat}
\bibliography{heat-flow.bib}

\end{document}
